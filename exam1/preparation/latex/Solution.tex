\documentclass[dvipdfmx]{jreport}

%--- ここを追加 ---
\usepackage[
  top=25mm,        % 上余白
  bottom=25mm,     % 下余白
  left=25mm,       % 左余白
  right=25mm,      % 右余白
  headheight=14pt, % ヘッダ行の高さ(jreport はヘッダを出すので余裕を持たせる)
  mag=1000         % 拡大率(原寸)
]{geometry}
%-----------------
\begin{document}


\section*{\centering アルゴリズムとデータ構造 前期期末試験 練習問題}
\section*{\normalsize\centering2025年7月9日版(微修正あれば更新します)}
試験対策向けの問題セットです。教科書等の内容で、授業の内容に沿って捻り出した問題集です。

\section*{情報工学レクチャーシリーズ アルゴリズムとデータ構造}
\section*{1 章 アルゴリズムの考え方、計算量の概念}
(1)アルゴリズムの定義を示せ。\\
(2)アルゴリズムを比較するための基準をなんと呼ぶか。\\
(3)教科書の p.2 問題 1.1 に書かれている n 桁の整数が 3 の倍数かどうかの判定について、数学的に判定する観点とコンピュータで判定する際の観点を比較して、問題の質の違いを説明せよ。\\
(4)コンピュータの計算資源の観点からアルゴリズムの性能を比較する際の評価指標を 2 点挙げよ。\\
(5)時間計算量について、アルゴリズムの性能を比較する際の観点を 3 点あげよ。\\
(6)アルゴリズムの時間計算量の性能を比較する際に、用いられる評価指標は何か。\\
(7)オーダ記法が使われる理由を説明せよ。\\
(8)教科書 p.4 アルゴリズム 1.3 について、不良品のボールを見つけるまでの時間の回数を求めよ。\\
(9)教科書 p.5 表 1.1 について、テニスボールの数が 10 から 100000 まで増加しても、最大の実行時間が約 170 秒で収まる理由を説明せよ。\\
(10)教科書 p.6 アルゴリズム $\mathrm{A} \cdot \mathrm{B} \cdot \mathrm{C}$ について、オーダ記法の時間計算量を示せ。\\
(11)教科書 p.7 中段に記載がある時間計算量について、大小関係を示せ。\\
(12)教科書 p.8 表 1.4 について、入力サイズ $n$ が大きくなると、アルゴリズムの実行時間はどのような観点 で比較できるか。\\
(13)教科書 p.10 アルゴリズム 1.4 について、最悪時間計算量を求めよ。\\
(14)教科書 p.10 アルゴリズム 1.5 について、最悪時間計算量の考え方を示せ。

\section*{2 章 配列、スタック、キュー、リスト}
(1)教科書 p.14 アルゴリズム 2.1 について、配列からデータを削除する場合の最悪時間計算量を求めよ。\\
(2)教科書 p.14 アルゴリズム 2.1 について、配列からデータを追加する場合の最良時間計算量、最悪時間計算量を求めよ。\\
(3)教科書 p.16 アルゴリズム 2.2 について、連結リストの考え方を説明せよ。\\
(4)教科書 p.16 アルゴリズム 2.2 について、連結リストを実現するためのデータ構造の考え方を説明せよ。\\
(5)連結リストのデータの追加•削除にについて、計算量の観点から特徴を説明せよ。\\
(6)スタックのデータ構造としての特徴を、計算量の観点から説明せよ。\\
(7)スタックの実装について、基本的な考え方を説明せよ。\\
(8)キューのデータ構造としての特徴を説明せよ。\\
(9)キューの実装について、基本的な考え方を計算量の観点から説明せよ。\\
(10)プログラミング語を用いてスタックを実装する場合の注意点を 1 点挙げて説明せよ。\\
(11)プログラミング語を用いてキューを実装する場合の注意点を 1 点挙げて説明せよ。

\section*{3章木}
(1)データ構造としての 2 分木の特徴を説明せよ。\\
(2)教科書 p.27 性質 3.1 について、計算量が成立することを確認せよ。\\
(3)教科書 p.27 性質 3.2 について、計算量が成立することを確認せよ。\\
(4)教科書 p.27 性質 3.3 について、計算量が成立することを確認せよ。\\
(5)教科書 p.27 性質 3.3 について、木の接点数の総和が $2^{h}-1$ が成立することを確認せよ。\\
(6)教科書 p.27 性質 3.4 について、計算量が成立することを確認せよ。\\
(7)教科書 p.28 の図 3.4 、図 3.5 について、 2 分木が配列で表現できることを説明せよ。\\
(8)教科書 p.29 問題 3.1 について、 $n$ 分後の試験管中の細胞の数を求める式を求めよ。\\
(9)教科書 p.29 問題 3.1 について、 $n$ 分後の試験管中の細胞の数を求めるアルゴリズムを示せ。\\
(10)教科書 p.31 アルゴリズム 3.3 について、和の計算を再帰木を用いて求めることができることを説明 せよ。\\
(11)教科書 p.31 アルゴリズム 3.3 について、和の計算を再帰木を用いて求める場合の計算量の考え方を説明せよ。\\
(12)教科書 p.32 アルゴリズム 3.4 について、和の計算を 2 項の数列に分けて計算する方法場合の、漸化式 を示せ。\\
(13)教科書 p.33 図 3.8 について、アルゴリズム 3.4 の時間計算量を求めよ。

\section*{4章2分探索、ハッシュ法}
(1)アルゴリズムにおける探索の定義を説明せよ。\\
(2)教科書 p.37 アルゴリズム 4.1 線形探索について、最良時間計算量、最悪時間計算量の考え方を説明 せよ。\\
(3)教科書 p.40 アルゴリズム 4.22 分探索について、アルゴリズムの考え方を説明せよ。\\
(4)教科書 p.40 アルゴリズム 4.22 分探索について、最悪時間計算量を求めよ。\\
(5)教科書 p.41 ハッシュ方について、データの衝突を無視する場合、ハッシュ法と線形探索• 2 分探索と の違いについて、目的のデータが見つかるまでの時間計算量の観点から違いを説明せよ。\\
(6)教科書 p.42 図 4.5 について、ハッシュ法の探索例を確認せよ。\\
(7)ハッシュ法のハッシュ関数に求められる特性を検討せよ。\\
(8)教科書 p.44 図 4.6 について、ハッシュ法の探索例を確認せよ。\\
(9)ハッシュ法において、探索した結果、データが格納されている場合にはどのような回避方法があるか、 2点挙げて説明せよ。\\
(10)教科書 p.45 ハッシュ法の性質に関して、ハッシュ法を配列を用いて実現する場合に成立する性質につ いて、説明せよ。\\
(11)教科書 p.46 データ探索における実行時間の比較について説明せよ。

\section*{5章挿入ソート、選択ソート、ヒープソート}
(1)アルゴリズムにおけるソートとはどのような操作か。\\
(2)教科書 p.50 アルゴリズム 5.1 について、選択ソートの考え方を説明せよ。\\
(3)教科書 p.50 アルゴリズム 5.1 について、選択ソートの最悪時間計算量を求めよ。\\
(4)教科書 p.51 アルゴリズム 5.2 について、挿入ソートの考え方を説明せよ。\\
(5)教科書 p.51 アルゴリズム 5.2 について、挿入ソートの最悪時間計算量を求めよ。\\
(6)ヒープソートを構成する 2 つの操作を挙げて、説明せよ。\\
(7)ヒープソートはどのようなデータ構造の特徴を用いたソート方法であるか説明せよ。\\
(8)ヒープソートに用いられるヒープに求められるデータ構造としての性質を説明せよ。\\
(9)教科書 p.56 図 5.7 について、ヒープへのデータの追加の手順を確認せよ。\\
(10)教科書 p.56 図 5.8 について、ヒープへのデータの追加の手順を確認せよ。\\
(11)ヒープソートにおける、ヒープから最大値の取り出し時の手順について説明せよ。\\
(12)教科書 p.57 図 5.9 について、ヒープから最大値の取り出し時の手順について説明せよ。\\
(13)教科書 p.58 図 5.10 について、ヒープを表す配列に対するデータの取り出しの手順について確認せよ。\\
(14)教科書 p.59 アルゴリズム 5.5 について、ヒープソートの手順の構成を説明せよ。\\
(15)ヒープソートの最悪時間計算量の求め方の考え方を説明せよ。

\section*{6 章 クイックソート}
(1)教科書 p.62 のクイックソートについて、ソートの考え方を説明せよ。\\
(2)教科書 p.63 図 6.1 について、クイックソートのソートの考え方を確認せよ。\\
(3)教科書 p.64 図 6.2 について、クイックソートの再帰木によるソートの考え方を説明せよ。\\
(4)教科書 p.64 図 6.3 について、関数 partitionの実行例を確認せよ。\\
(5)教科書 p.65 アルゴリズム 6.1 について、クイックソートの手順が 6.1 の流れになる理由を説明せよ。\\
(6)クイックソートの関数 partition に求められる処理は何か説明せよ。\\
(7)教科書 p.66 図 6.4 の関数 partitionの実行例を確認せよ。\\
(8)教科書 p.66 アルゴリズム 6.2 の手順と図 6.4 を対比させながら、分割操作の考え方を確認せよ。\\
(9)クイックソートにおいて、最悪時間計算量が $\mathcal{O}\left(n^{2}\right)$ となる入力例はどのようなものか考察せよ。\\
(10)クイックソートにおいて、最悪時間計算量の求め方の考え方を説明せよ。\\
(11)教科書 pp.67-68を参考に、クイックソートの最悪時間計算量を求めよ。\\
(12)クイックソートにおいて、最良時間計算量で動作する場合、どのような再帰木が構成されるか説明せよ。\\
(13)教科書 pp.67-68 を参考に、クイックソートの最良時間計算量を求めよ。\\
(14)教科書 p.70 図 6.7 について、ソートアルゴリズムの性能比較について、説明せよ。\\
(15)教科書 p.71 図 6.8 について、ソートに求められる安定性とはどのようなものか説明せよ。

\section*{1-6 章の章末の演習問題}
(1) 1 章 1.1 (1.2、1.3は取り組む価値あり)\\
(2) 2 章\\
(3) 3 章\\
(4) 4 章\\
(5) 5 章\\
(6) 6 章

\section*{新•明解 C 言語で学ぶアルゴリズムとデータ構造}
\section*{3章 探索}
(1)p.90 線形探索の考え方、実装を確認せよ。\\
(2)p.962 分探索の考え方、実装を確認せよ。\\
(3)p.114 ハッシュ法の考え方、実装(チェイン法、オープンアドレス法)を確認せよ。

\section*{4 章 スタック、キュー}
(1)p.146 スタックの考え方、実装を確認確認せよ。\\
(2)p.156 キューの考え方、実装を確認確認せよ。

\section*{5章再帰}
再帰アルゴリズムの解析、 8 王妃問題は除く。\\
(1)pp.174-175 再帰の考え方、例を確認せよ。\\
(2)p.188 ハノイの塔について、解法、例、実装を確認せよ。

\section*{6 章 ソート}
(1)p.218 選択ソートの考え方、実装を確認せよ。\\
(2)p.220 挿入ソートの考え方、実装を確認せよ。\\
(3)p.230 クイックソートの考え方、実装を確認せよ。\\
(4)p.256 ヒープソートの考え方、実装を確認せよ。

\section*{8 章 線形リスト}
(1)p.309 ポインタによる線形リストの考え方、実装を確認せよ。

\section*{9章 木構造、2 分探索木}
(1)p.364 木構造の考え方を確認せよ。\\
(2)p.3682 分木、2分探索木の考え方、実装を確認せよ。


\end{document}